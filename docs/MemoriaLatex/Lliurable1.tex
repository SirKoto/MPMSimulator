\documentclass[a4paper]{report}

\usepackage{amsmath}
\usepackage{graphicx}

\usepackage{anyfontsize}

\usepackage{fancyhdr}
\pagestyle{fancy}

\usepackage{titlesec}
\titleformat{\chapter}[block]
{\normalfont\huge\bfseries}{\thechapter.}{1em}{\Huge}
\titlespacing*{\chapter}{0pt}{-19pt}{19pt}

\usepackage[nottoc,numbib]{tocbibind}


\renewcommand{\familydefault}{\sfdefault} % sans serif default



\setcounter{tocdepth}{1} % Show sections
\setcounter{tocdepth}{2} % + subsections
%\setcounter{tocdepth}{3} % + subsubsections
%\setcounter{tocdepth}{4} % + paragraphs
%\setcounter{tocdepth}{5} % + subparagraphs

\title{Aplicació del Material Point Method (MPM) a la deformació d'objectes}

\author{Martín Garcia, Pol}
\date{\parbox{\linewidth}{\centering%
		\today\endgraf\bigskip
		Director: Susín Sánchez, Toni\endgraf \medskip
		Especialitat Computació \endgraf
		Facultat d'Informàtica de Barcelona}}

\lhead{\includegraphics[width=5cm]{images/logo-fib.png}}
\rhead{\fontsize{5}{6}\selectfont{\textbf{MPM a la deformació d'objectes} \\
			Lliurament 1 GEP\\}}
		
		
\begin{document}
	\pagenumbering{gobble}
	\maketitle
	\newpage
	

	
	\tableofcontents
	\newpage
	
	\pagenumbering{arabic}	
	
	\chapter{Introducció}
	Aquest treball de fí de grau ha estat centrat en la simulació de fluids, una temàtica molt amplia que tot i que conceptualment sembla trivial, és molt complexa. Molts detalls i decisions s'han de tenir en compte a l'hora d'implementar un simulador d'aquestes característiques, com la representació del fluid, si aquest es compressible, elàstic, col·lisions, etc. \newline
	Per a poder presentar el treball correctament, aquest document també serveix com a introducció a la simulació de fluids sense assumir coneixements previs en aquest aspecte.\newline
	I finalment també es busca mostrar com materials sòlids-elàstics es poden representar i simular com a fluids.\newline 

	\section{Context}
	La simulació de fluids, o dinàmiques de fluids computacionals (CFD's), és una disciplina que mitjançant tècniques d'analisi numéric busca resoldre problemes que impliquen, d'alguna manera, algun tipus de fluctuació o interacció amb fluids . \newline
	La base de qualsevol simulador de fluids són les equacions de Navier-Stokes, que descriuen la mecànica d'aquests; però necessitem molt més que les equacions per a poder implementar aquest programa, doncs també son necessaris coneixements de software i hardware per a una implementació eficient, tècniques de gràfics per computador per a una adequada visualització, i coneixements matemàtics d'àlgebra lineal que ja s'han vist durant l'educació oferida per la Facultat d'Informàtica de Barcelona, sobretot en la branca de computació. Tot i així, l'abast del projecte ens obliga a sortir de l'àmbit de coneixements d'un enginyer informàtic, i integrar múltiples conceptes matemàtics de diferents caires, que seran explicats degudament. \newline
	En el nostre cas les matemàtiques ens ajudaran a definir el comportament i les limitacions del nostre simulador, que a partir d'ara anomenaré \textit{solver}, per a poder configurar-lo amb coneixement a posteriori, i sempre ser conscients de l'estat sistema que s'està processant. \newline
	
	\subsection{Conceptes bàsics}
	Abans d'entrar en detalls, hi ha uns coneixements bàsics que s'han de tenir en compte degut a l'amplia projecció dels simuladors de fluids.
	\subsubsection[Tipus de solvers]{Tipus de solvers} 
	Podem diferenciar els tipus de solvers en 3 subgrups, que varien intrínsecament en el mateix concepte de la representació del fluid. Aquesta representació marcarà la manera d'interactuar i tractar el fluid tant amb ell mateix, com amb cossos externs.
	
	\paragraph[Graella]{Solvers en Graella} Podem representar el fluid en un instant de temps concret com a una magnitud en un punt d'una graella, identificant la quantitat de fluid en aquella posició en un moment concret; per altra banda, cada cel·la de la graella té alguna representació de direcció i magnitud de moviment del fluid. \newline
	A més a més, podem emmagatzemar altra informació a cada posició de la graella, o usar les dades ja guardades per a processar l'estat del sistema en un instant de temps posterior, tenint en compte que a cada cel·la només hi pot haver una quantitat màxima de fluid (volum màxim) per a que no sigui un fluid comprimible, i així provocar una dissipació d'aquest i el conseqüent moviment.\newline
	En conjunt creem un espai acotat per les dimensions de la graella, on el fluid es mou de manera quantitativa a traves de les diferents cel·les d'acord amb la pròpia informació d'aquestes. \newline
	Aquesta representació es coneix com simulador Eulerià.
	\paragraph[Partícules]{Solvers de Partícules} Podem representar un petit volum de fluid com a una partícula en una posició a un espai indeterminat en un instant de temps, de manera que sempre representa la mateixa quantitat de fluid, i a més guardem tota la informació de moviment (força, velocitat,...) a un nivell molt concret. \newline
	És important entendre que les partícules no representen el fluid a nivell de molècules o àtoms, sinó que cada partícula representa una porció continua del material o fluid, o un subconjunt del domini físic a simular. 
	\newline
	Aquesta representació permet, per exemple, barrejar dos fluids mantenint alhora les seves propietats separades, o simular interaccions entre partícules o sòlids externs amb molt més nivell de detall.
	També cal dir, que aquest mètode dificulta la tasca de mantenir un volum constant del fluid, i sobretot necessita computar interaccions entre totes les partícules, comput que resulta costos. \newline
	Els simuladors de partícules són coneguts com a Lagrangians.
	\paragraph[Híbrids]{Solvers Híbrids} Si mesclem els dos conceptes anteriors, podem obtenir un simulador que, podem dir, rep el millor dels dos mètodes.\newline
	Per una banda, representem el fluid com partícules a cada instant de temps amb totes les seves propietats; i per altra banda el comput d'interaccions entre aquestes el gestionem mitjançant una graella (la qual defineix l'espai del sistema) a on hi traspassem les característiques de les partícules en determinades zones, de manera que la computació de col·lisions i/o interaccions és molt més eficient i acotada.   
	
	\subsubsection{Propietats del fluid} 
	Un fluid sempre ha de tenir representades, en alguna determinada estructura o directament en l'algorisme, un seguit de propietats que el defineixen i n'especifiquen el comportament. Cal dir, que només és necessari usar aquelles propietats que el nostre model té en compte.
	\paragraph{Volum} La quantitat de fluid que volem simular té un impacte directe en el nostre \textit{solver}, i depenent de la representació del fluid aquest concepte afectara de manera diferent; per exemple en simuladors de graella massa fluid pot saturar el model, i en de partícules hem de controlar la proximitat entre partícules defineix el volum. 
	\paragraph{Massa} Depenent del model a representar, ens pot interessar la massa del fluid (representada, per exemple, en densitat) per a tractar la interacció amb forces externes (col·lisions o gravetat).
	\paragraph{Elasticitat} L'elasticitat es l'habilitat d'un cos o material de recuperar la seva forma original després d'una deformació, i alhora resistir-se a aquestes forces de deformació. És modelen amb el model i la llei de Hook, que estipula que la força de retorn és proporcional a l'extensió de la deformació.
	\paragraph{Viscositat} Molt semblant a l'elasticitat, la viscositat d'un fluid és la resistència que té a deformar-se o, informalment, a escampar-se. Es mesura en pascals per segon ($\frac{Pa}{s}$ o $\frac{kg}{m \cdot s}$) i és representa amb $\mu$ o $\eta$. És molt comú representar aquesta magnitud en relació amb la densitat $\rho$ de manera $\nu = \frac{\mu}{\rho}$ per a simplificar expressions.
	\paragraph{Plasticitat} Si l'elasticitat defineix la possible deformació d'un cos, la plasticitat marca el punt de no retorn d'una deformació, en que les forces que busquen el retorn de l'objecte a l'estat original son permanentment degradades. La idea de modelar aquesta característica, i el pioner treball de Terzopoulos et al. \cite{Terzopoulos:1988:MID:378456.378522} el 1988, la simulació de materials topològicament canviants és una popular àrea d'estudi a gràfics per computador, on es poden tractar fets com la fractura (límit de la plasticitat), i el tallament d'objectes deformables. Totes aquestes interaccions s'anomenen inelàstiques, ja que trenquen l'elasticitat del cos.
	\paragraph{Viscoelasticitat} Propietat dels materials que son viscosos i elàstics alhora en el moment de deformar-se. La viscositat fa que, conforme el temps avança, el material sofreix una tensió que li provoca deformació, de manera que tota deformació elàstica sempre comporta una pèrdua d'energia degut a la viscositat, produint una deformació plàstica.
	\subsubsection[MPM]{Material Point Method}
	Un dels metodes, ara, que han resurgit per a la simulació de fluids és el conegut Material Point Method proposat per Sulsky \cite{Sulsky1995}, basat en particules Lagrangianes i una graella Euleriana de rerefons. Aquest mètode és usat donat la seva alta eficiència de comput, i la possibilitat de detallar la seva precisió amb la mida de la graella. \newline
	Aquesta tecnica intenta emmagatzemar les dades de les partícules de manera interpolada en les ce·les més properes, du a terme els càlculs en la graella, i finalment es fa la interpolació de manera inversa i és fa el desplaçament de la partícula. \newline
	Tot i això, MPM té alguns problemes ja que degut a la simplificació que es du a terme a la graella, no som capaços de representar velocitat discontinues, o col·lisions molt concretes. 
	
	\subsection{Formulació del problema}
	Podríem considerar que hi ha dos tipus de simulacions: les que tenen com a objectiu emular la realitat, i les que busquen obtenir una animació visualment realista.\newline
	Les simulacions que imiten la realitat tendeixen a requerir de moltíssima precisió i algorismes poc optimizables, que acostumen a requerir ser processats en un supercomputador per obtenir resultats adequats en un temps raonable. \newline
	Per altra banda, la simulació amb objectius només visuals (com és en nostre cas) intenta ser ràpida i, sobretot, configurable per a poder triar quins resultats es volen aconseguir. Encara que no s'intenti emular la realitat al cent per cent, és important imitar-la al màxim acceptant generalitzacions i truncaments per a accelerar la velocitat de computació; fins i tot arribar a tenir una simulació en temps real.\newline
	Degut a que és busca trobar una fidel i ràpida aproximació a la realitat, no existeix una solució analítica (ni general ni específica) per a produir aquestes simulacions. També s'ha de tenir en compte quina és la versatilitat de la simulació, doncs el propi algorisme implementat limitarà les propietats del fluid o solid. Finalment també hi ha la possibilitat, i el problema, de fer interaccionar el fluid a simular amb algun cos extern (tant amb moviment com sense) i la transmissió de forces entre aquests. \newline
	D'aquesta manera, aquest treball pretén estudiar com plantejar-se solucions per a implementar les qüestions anteriors, i aplicar-les alhora per a provar les limitacions i habilitats de cadascuna, en un determinat sistema; i també servir com a base introductòria a aquesta temàtica tant dispersa i tècnica.
	
	\subsection{Actors implicats}
	En aquesta secció és definiran els actors implicats, de manera directa i indirecta, en el projecte.
	\paragraph{Director} Figura essencial pel correcte desenvolupament del projecte, i màxim responsable en la guia i consell del desenvolupador del treball. La seva acció és clau per determinar errors en el projecte, tant de tipus proposicional com executius. En aquest cas el director és Toni Susín Sánchez, del departament de Matemàtiques Aplicades de la UPC, i cap del \textit{Dynamic Simulation Lab} inclòs en el grup de recerca ViRVIG.
	\paragraph{Desenvolupador} La persona a càrrec de dur a terme la recerca, documentació i implementació del software requerit, així com de redactar la memòria del projecte sota la guia del director. També recau sobre aquest actor la gestió del projecte, i és en ultima instancia la persona en cap de complir els terminis del treball. Aquest rol recau sobre Pol Martín Garcia, alumne de la FIB de la branca de computació.
	\paragraph{Entorn} El centre de treball ViRVIG (Visualització, Realitat Virtual i Interacció Gràfica) ha facilitat un entorn de treball per a poder usar i probar el projecte amb la potencia de comput necessària, i aprovisionat d'un clima de treball per al correcte desenvolupament del projecte.
	\paragraph{Usuaris} Aquest treball busca crear una dissertació i ampliació del simuladors d'avui dia, i com a tal no tindrà usuaris directes. Qualsevol avanç en \textit{solvers} de tot tipus, però, té un impacte parcial en la investigació i implementació de noves tecniques d'animació; com per exemple en efectes especials de pel·lícules, com a \textit{Big Hero 6}, \textit{Zootopia}, o inclús \textit{Frozen} de Disney\cite{Stomakhin} per a simular la neu.  
	\paragraph{Beneficiaris} Com que aquest projecte no creara un producte, com s'ha dit abans, tampoc hi ha beneficiaris directes. No obstant això, altres estudiants d'enginyeria informàtica o matemàtiques poden trobar aquest projecte interessant com a base d'aprenentatge, i punt de partida per a futurs projectes centrats en l'àmbit de simulació de fluids en MPM. 
	
	\section{Justificació}
	Aquesta temàtica és una amplia àrea d'estudi des de fa diverses dècades, que s'ha anat dividint en múltiples tècniques i cada branca d'estudi va aprofundint en paral·lel de la resta.
	\newline
	En aquest treball estudiarem i implementarem un \textit{solver} MPM, que neix de la generalització de \textit{Particle in Cell} (PIC) i \textit{Fluid Implicit Particle Method} (FLIP) \cite{Sulsky1995}. Hi ha hagut molt numerosos i extensos treballs en FLIP en els darrers anys \cite{Bridson2018,Zhu2005}, mentre que MPM ha sigut realment introduït i començat a estudiar recentment, degut a que al contrari que FLIP aquest pot tractar amb fluids i sòlids comprimibles.
	\newline
	D'aquesta manera, MPM s'esta començant a usar i a investigar tant per simular molt diferents tipus d'interaccions, com s'ha dit abans ressalta la simulació de neu de Disney \cite{Stomakhin}, interacció entre o amb diversos materials deformables\cite{Hegemann2013}, i molts demes. Podeu trobar més informació de treballs en MPM a \cite{Jiang2016}.
	\newline
	Just molt recentment ha sorgit una nova tècnica per a resoldre MPM, anomenada MLS-MPM (\textit{Moving Least Squares Material Point Method})\cite{hu2018mlsmpmcpic} que, generalment, busca millors aproximacions a la graella gràcies a una B-spline\footnote{Funció corba. S'en parlarà més endavant en el treball de que són.} calibrada amb MLS. 
	\newline
	D'aquesta manera, i amb base dels nous desenvolupaments de Yuanming Hu et. al. a \cite{hu2018mlsmpmcpic} i usant com a base \cite{Hu} del mateix, estudiarem aquesta tècnica i possibles diferents interpretacions del model. És a dir: adaptarem la solució de Yuanming a unes necessitats personals i investigar com reacciona el sistema a unes noves condicions.
	
	\section{Abast}
	En aquesta secció definirem quines són les fites del projecte, i la seva estructura progressiva.
	\subsection{Objectius}
	\paragraph{Aprenentatge de nous conceptes} Donat que el treball requereix de coneixements allunyats dels adquirits en una enginyeria informàtica, cal una cerca intensiva d'informació i aprenentatge de mètodes numèrics, càlcul, i àlgebra lineal. Tots aquests conceptes són necessaris per entendre els diferents models d'objectes i simuladors a implementar. \newline
	En concret, existeixen alguns subtemes que mereixen una menció especial com a sub-objectius:
	\subparagraph[Navier-Stokes]{$\cdot$ Les equacions de Navier-Stokes} Aplicant la segona llei de Newton al moviment del fluid, amb algunes assumpcions de les forces internes (anomenades estrès), s'obtenen les equacions de Navier-Stokes; conegudes per la seva complexitat i que ja de per si, obren una interessant àrea d'estudi de possibles implicacions i aplicacions.
	\newline
	És important entendre aquestes equacions, i saber interpretar-les, doncs descriuran el nostre sistema.
	\subparagraph[Càlcul vectorial]{$\cdot$ Càlcul vectorial} Fins a un cert nivell, és important i necessari conèixer diferenciació i integració de camps vectorials, doncs tractarem amb camps de partícules en aquest projecte.
	\subparagraph[Tècniques de solvers]{$\cdot$ Tècniques de solvers} Finalment, amb els coneixements teòrics necessaris, s'ha d'aprendre el funcionament intern i els models que permeten simular els diferents tipus de \textit{solvers} creats, en concret el de MLS-MPM\cite{hu2018mlsmpmcpic}.
	
	\paragraph{Implementació d'un solver bidimensional} Per a una primera implementació, s'implementarà un simulador bidimensional de fluids de partícules, seguint\cite{Hu,hu2018mlsmpmcpic} i programant la base de tot el projecte a una escala menor, per a fer proves i trobar errors. La idea és usar aquesta primera fase com a indicador, i prendre decisions d'acord amb els resultats obtinguts per a millorar el resultat final, i evitar errors. \newline
	L'objectiu d'aquesta fase és el refinament de l'entorn de treball per a una posterior fase molt més complexa, i evitar nombrosos possibles errors, estudiar inconvenients que puguin sortir i, sobretot, com a guia d'aprenentatge progressiu de manera simplificada. 
	
	\paragraph{Millora del solver a la tercera dimensió} El canvi de dimensió és un salt difícil, i essencial en aquest projecte, doncs tot simulador gràfic per a resultats realistes requereix d'aquesta tercera dimensió. Serà necessari redimensionar la interfície gràfica, les estructures de dades, i el propi algorisme.
	\newline
	A partir d'aquesta fita, és quan s'inicien els tests i objectius secundaris.
	\subparagraph[Interacció viscoelàstica]{$\cdot$ Interacció entre diversos materials viscoelàstics diferents} Afegir suport per a poder mantenir en el simulador alhora diversos fluids de diferents característiques cadascun.
	\subparagraph[Sistema de físiques]{$\cdot$ Sistema de físiques} Implementar un sistema d'interacció d'un cos solid en moviment amb el solver, per a aconseguir simular colisions amb un, o diversos, elements externs al simulador; com per exemple una roda amb sorra.
	
	\paragraph{Altres objectius} En la mesura del possible, s'intentarà dur a terme els següents objectius, que no son necessaris pel correcte desenvolupament del projecte, però el completen i ajuden a veure com tractar i solucionar altres problemes inherents en qualsevol sistema programat.
	\subparagraph[Eficiència]{$\cdot$ Eficiència} Sobretot en el \textit{solver} tridimensional, la eficiència i velocitat del programa és un problema ja que, aproximadament, el comput de cada fotograma serà major a un minut segons els articles de MPM; per tant sempre que és pugui s'ha d'implementar el simulador sent conscients de l'arquitectura i de l'eficiència del codi.
	\subparagraph[Paral·lelisme]{$\cdot$ Paral·lelisme} Aconseguir que el codi funcioni de manera paral·lela pot suposar un increment de la velocitat de computació molt important. Així doncs, qualsevol oportunitat de paral·lelitzar el codi s'haurà d'aprofitar.
	
	
	\subsection{Requisits}
	Podem separar els requisits del software a implementar en funcionals, i no funcionals. Els primers representen funcionalitats especifiques que el nostre sistema haurà de complir, i el segon especifica criteris per jutjar els comportaments específics del mateix.
	
	\subsubsection{Requisits funcionals}
	\paragraph{Simulació d'un sistema personalitzable} Amb modificació del nombre, tipus, i posició de les partícules; canvi del medi (gravetat, mida de la graella, etc ...). Cal valorar si han de ser paràmetres modificables entre execucions.
	\paragraph{Exportació de la simulació} Exportació dels fotogrames produits pel solver, o be en arxius binaris per al seu posterior tractament, o be en imatges PNG (\textit{Portable Network Graphics}).
	\paragraph{Simulació d'objectes viscoelàstics} Importació de models de triangles, i simular-ne el seu comportament en unes condicions especificades com a materials elàstics o viscoelàstics.
	
	\subsubsection{Requisits no funcionals} 
	\paragraph{Velocitat de computació} La velocitat del comput ha d'estar justificada i ser adient; ha de ser linealment depenent al nombre de partícules o a la mida de la graella. 
	\paragraph{Estabilitat de la simulació} Les solucions del \textit{solver} han de ser estables, i no descontrolar-se; en cas de ser impossible evitar l'inestabilitat, aquest fet estarà documentat i estudiat, marcant els límits del simulador.
	
	
	\subsection{Obstacles i riscos}
	Per evitar imprevistos en el transcurs del projecte, analitzarem els obstacles que poden sorgir durant aquest procediment i, sobretot, com minimitzar el seu impacte.
	\subsubsection{Errors en el codi}
	És inevitable cometre petits errors en l'implementació d'un software, que a vegades poden passar desapercebuts, i altres vegades poden trencar totalment el programa.\newline
	Donat que un simulador no pot ser subjecte a jocs de proves, serà necessari provar manualment cada nova característica afegida, i comprovar que totes les anteriors propietats segueixen funcionant adequadament. D'aquesta manera podrem minimitzar els riscos de perpetuar un error durant les diverses versions del software.
	\subsubsection{Problemes o insuficiència de hardware}
	Els ordinadors poden tenir problemes, els discos poden fallar, la memòria RAM pot deixar de funcionar, els arxius es poden corrompre, etc... El qual suposa una pèrdua de temps important a curt termini, per a arreglar el hardware espatllat i recuperar les dades perdudes. \newline
	Per evitar l'impacte d'aquest problema en la mesura de tot el possible, el projecte i la memòria d'aquest és guardaran per versions en un repositori a la xarxa, més concretament a \textit{GitHub}. A més a més, la memòria és guardarà en paral·lel al \textit{OneDrive} de Microsoft.
	\newline
	Per altra banda, durant les etapes més avançades del projecte el temps de simulació incrementarà molt, cosa que provocarà que per provar els canvis siguin necessàries hores. \newline
	Per a evitar perdre temps sense efectuar res, els tests és provaran en una màquina diferent en la que es treballa, amb molta més potencia (com un dels ordinadors del ViRVIG) per obtenir els resultats el més aviat possible, sense entorpir el correcte progrés.
	\subsubsection{Errors de comprensió dels models matemàtics}
	Degut a la complexitat de les matemàtiques que hi ha darrere d'aquest projecte, i al fet que aquest es un projecte d'un informàtic, és possible errar en l'enteniment de nombrosos conceptes en, per exemple, els models elàstics dels objectes, el procediment del simulador, o en simples formalismes d'expressivitat matemàtica. \newline
	\subsubsection{Problemes temporals}
	Les limitacions temporals son inherents a tot projecte de fi de grau, i aquest no n'es una excepció. Si hi ha una mala gestió del temps, o algun tipus d'imprevist, podem trobar-nos amb la incapabilitat de complir tots els objectius del treball a temps. \newline
	Per a aquest motiu és important escollir i seguir alguna metodologia, donada una planificació per a dur a terme el projecte. D'aquesta manera podem ser molt més conscients del avenç que s'està fent, i de ser necessari poder realitzar a temps modificacions a l'abast del mateix.
	
	\section{Metodologia i rigor}
	En qualsevol projecte de mínima magnitud, s'ha de triar i adherir-se a un sistema organitzatiu concret. D'aquesta manera podem tenir una visió molt més acurada del projecte, i mantenir un ordre en aquest; a més de tenir sempre procediment a seguir en cas de qualsevol inconvenient.
	\subsection{Metodologia}
	La metodologia emprada en el projecte per a la seva organització i concepció és la \textit{Kanban} \cite{Gould2012}. Aquesta és una metodologia àgil molt semblant a \textit{Scrum}, però dissenyada per a mantenir un flux constant de resultats evitant discretitzar el projecte en \textit{sprints}. Aquest mètode aporta molta flexibilitat i permet focalitzar el treball en només el que es considera essencial.
	\newline
	La principal idea darrera d'aquest mètode és mantenir sempre alguna tasca en procés, de manera que quan s'ha acabat quelcom, la següent tasca de més prioritat entra en acció. A més a més, s'ha d'evitar mantenir més d'una tasca en procés alhora, limitant la quantitat de \textit{work in progress} al mínim, i centrar-se en el més important en aquell moment.
	\newline
	En cas de tenir qualsevol inconvenient, aquesta metodologia és molt responsiva als canvis i ens permet, ràpidament, corregir qualsevol aspecte mancant, alhora que és valora la seva prioritat amb el director del projecte.
	
	\subsection{Eines de seguiment}
	Per al correcte seguiment del projecte, el seu estat i les seves versions s'usen diverses eines. Primerament és important tenir una visió global del que es porta fet, i el que queda per fer; per a això s'usa el mètode de \textit{Bullet Journaling} per a ser conscients en tot moment de l'estat del projecte, a un nivell personal. \newline
	A nivell de codi, un historial de versions, organitzat adequadament per funcionalitats, és emmagatzemat a \textit{GitHub} juntament amb les diverses versions de la memòria del projecte. Amb aquesta eina podem tenir una visió ordenada del procés evolutiu del treball, veient les funcionalitats per separat.\newline
	Finalment, també es duran a terme reunions periòdiques amb el director on es revisaran, entre altres coses, els avanços fets des de l'ultima reunió, l'estat del projecte i dubtes i qüestions sorgides. Donada la natura del projecte, en aquestes reunions es duran a terme explicacions teòriques de conceptes no vistos a la FIB per al correcte desenvolupament del projecte.
	\newline
	En cas de sorgir problemes o imprevistos, es contactara al director i, de ser necessari, es creara un pla de contingència o una ruta a seguir per a solucionar qualsevol inconvenient. 
	 
	
	
	
	% REFERENCIES
	\newpage
	
	\renewcommand{\bibname}{Referencies}
	
	\bibliography{BibTex/cit} 
	\bibliographystyle{ieeetr}
	
\end{document}